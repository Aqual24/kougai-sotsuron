\documentclass[twocolumn,10pt,a4j]{ltjsarticle}
\usepackage{kougai}

\title{娯楽ゲームの教育的活用を推進するWebサイトの開発}
\author{1732008 五十嵐美結  指導教員 須田 宇宙 准教授}
\date{}

\begin{document}

\maketitle

\section{はじめに}\label{introduction}
%1章には,背景・問題点・目的を順番に書く.
%背景は,広く一般的な事柄を書いて,読む人に同意を抱かせつつ問題点につなぐ.
%問題点では,「〜という問題点がある」などのように,「問題」または「問題点」と言う単語を用いて,目的につなぐ.
%目的では,「そこで本研究では」から始めて,「〜を目的とする」で締める.
%以下は過去の卒業研究最終審査用の梗概の抜粋である.

%背景

近年,アクティブ・ラーニングとして授業活動にゲーミフィケーションといわれるゲームの娯楽性要素や,学習要素を盛り込んだシミュレーション等のゲーム(シリアスゲーム)を導入する動きが活発になってきている.
ゲーミフィケーションは楽しさ,目的意識,達成感の充実といったゲームの主要な要素を取り入れることによって授業への参加意欲や充実感の向上のために活用されている.


%問題点
一方でデジタルゲームのうち娯楽要素の多いゲームはゲーム依存症のイメージがあり,教育的なメリットは周知されておらず自宅での学習の妨げになるなどの問題がある.
またこの問題により保護者からプレイ制限をされることで,ゲームから得られる学習機会の損失になるという点がある.

%目的
そこで本研究では,学習を主目的としないデジタル娯楽ゲームの印象の改善とそれらの持つ教育的効果の周知を図るために,様々な娯楽ゲームの持つ教育的なメリットを明示するWebサイトを作成し,それにより学習機会があることを理解したかを保護者へのアンケート調査を行い評価することを目的とする.

\section{教育指向のゲームと娯楽ゲーム}

\ref{introduction}章で紹介したシリアスゲームはデジタルゲームの一種で主にコンピュータやタブレットなどを使用し,教育・医療・環境といった社会問題の解決を目的として,英語や情報技術分野等では実際に教育現場で活用されている.
このようなゲームは学習に使用するために開発されたゲームのであり教育効果が分かりやすく,保護者や教育者が導入しやすい.
一方で娯楽ゲームは娯楽を主目的に開発されたゲームであり,それから得られる教育効果が明示されていないため推奨されにくく,また依存症・ゲーム脳等のイメージが広まっていることから悪い影響があると思われプレイ制限に繋がってしまう.
ただゲーム技術が発展し多くの娯楽ゲームを教育活用する動きがあるのも事実である.\cite{tvgame}

そこで本研究では図\ref{fig:ゲーム一覧}に示す12本のゲームを扱い,教育効果を明示し娯楽ゲームにも学習機会があることを示した.

%本稿でのデジタルエンターテイメントゲームは家庭用ゲーム機やパソコン,タブレット,スマートフォン等でプレイするコンピュータゲームを指し,カテゴリはアクションやRPG,アドベンチャーなど多岐にわたる.シリアスゲームに挙げられるシミュレーションゲームも含まれるが,ここでは学習を目的としない娯楽指向のゲームについて示す.


\section{Webサイトとアンケートについて}
本研究では教育効果の明示による印象の変化を調査するため, Webサイトに掲載する娯楽ゲームは小中学生を対象にしたものに設定し,アンケートの対象者は小中学生の子を持つ保護者に設定した.
「あつまれどうぶつの森」の教育的メリットや教科は図\ref{fig:ゲーム一覧}右側のように設定した.
紹介したゲームは左側の12本で右側はソフトにつけたタグである.
記事の内容には紹介した娯楽ゲームの概要とタグ付けしたそれぞれの教育的なメリットについての説明をゲームの内容に沿って記述した.

アンケートは子の年齢やゲームのプレイ頻度,好き嫌い等の質問と,表\ref{table:anque}の項目を見る前と見た後について10人に質問した.
結果は表\ref{table:anque}のi~ⅳの項目について閲覧後は意見が好転傾向であったが,v,ⅵは熱中しうることや視力低下の懸念等で意見の変化は少なかった.ⅶは比較的デメリットが多いと考える人が4割だったが閲覧後2割に減少し,どれも同じくらいと考える人が7割になりメリットが多いと答える人も現れた.

\begin{figure}[h]
 \begin{center}
  \includegraphics[clip,width=95mm,height=55mm]{games.pdf}
 \end{center}
 \caption{ゲーム一覧と「あつまれどうぶつの森」のタグ付け例}
 \label{fig:ゲーム一覧}
\end{figure}

\begin{table}[h]
 \caption{アンケート項目}
 \label{table:anque}
 \small
 \centering
  \begin{tabular}{l}
  \hline
  子に与える影響について \\
  \hline
   ⅰ.勉強面 \\
   ⅱ.友人関係,コミュニケーション\\
   ⅲ.感性\\
   ⅳ.知識,教養 \\
   ⅴ.時間管理   \\
   ⅵ.健康面 \\
   ⅶ.読書・スポーツ等他の趣味とのメリットの比較 \\
   \hline
  \end{tabular}
\end{table}


%\section{Webサイト制作と評価方法}
%掲載するエンターテイメントゲームは様々な種類のものを用意し,それぞれのゲームの種類やプレイによって得られるだろう教育的メリットでジャンル分けしタグ付けを行う.タグは,創造性,協調性,社会理解,デジタル理解,自然理解などの他,ワード検索もできるようにする予定である.またタグの他に,授業の教科に関連付けを行い教科からも検索できるようにする.図\ref{fig:スクショ}のゲームの詳細ページを示す.①の部分は検索機能でタグやキーワードから検索ができる欄である.②の部分はゲームのスクリーンショットとゲームプレイによって得ることが期待される教育的メリットについての詳細や具体例を記述する.


\section{おわりに}
本研究では娯楽ゲームの教育的メリットの周知と印象の改善を図るWebサイトによりアンケートで教育的な効果があるという評価が得られた.
一方説明文がゲームをしない人にとって内容が想像しにくい点や根拠が薄い点があるため,今後は改善した記事内容にすることが望まれる.
%今後はこのような娯楽ゲームのメリットが広く認知され,記述や教育の発展に繋げられることを期待している.

\begin{thebibliography}{99}
\bibitem{gameanq} ASMARQ:``ゲームと子どもに関するアンケート調査'', \url{https://www.asmarq.co.jp/data/mr201409game/},2022/12/29参照
\bibitem{tvgame} 坂元章 :``21世紀はテレビゲーミング社会 -娯楽主導から有効利用ヘ-'', 特定非営利活動法人日本シミュレーション\&ゲーミング学会, 2000年10巻1号p.4-13,2000.
\end{thebibliography}

\end{document}
